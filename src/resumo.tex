\begin{resumo}
  
A capacidade de armazenamento de dados alcan�ada pelo atual desenvolvimento            tecnol�gico viabilizou a muitas empresas e institui��es registrar dados relativos aos  seus neg�cios em busca de obter melhores resultados e respostas de seus servi�os.      Atualmente, muitas dessas empresas e institui��es obt�m bons resultados com a          explora��o de suas bases de   dados. Um exemplo conhecido s�o os supermercados e o com�rcio em geral, que buscam extrair padr�es e tend�ncias dos dados de seus clientes  de modo a obter melhores vendas. Muito t�m  se trabalhado na explora��o de dados    comerciais e cient�ficos, mas pouco na explora��o de dados pol�ticos e sociais. Com o  intuito de aprofundar o estudo desse tipo de dados, em especial de dados eleitorais,   este trabalho utilizou representa��es visuais e t�cnicas de teoria de informa��o para  identificar padr�es e revelar quais fatores influenciam as elei��es brasileiras.  Apontou-se caracter�sticas e peculiaridades das elei��es em diversos n�veis de        granularidade, como em n�vel nacional e regional. Avaliou-se como os candidatos se relacionam no aspecto de financiamento de campanha e como este fator pode influenciar
os resultados. Essas an�lises foram viabilizadas pela constru��o de um robusto banco de dados que � respons�vel por centralizar toda a informa��o adquirida esparsamente em diversas fontes. Este banco de dados juntamente com as an�lises realizadas resultam nas principais contribui��es deste trabalho, de modo que podem ser utilizadas como ponto de partida para trabalhos futuros.

\end{resumo}
