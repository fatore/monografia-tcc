\documentclass[brazil,11pt,a4paper,twoside]{icmc}
% pdftex
%\includeonly{sdl-pr} %para compilar apenas o cap1

\usepackage[top=30mm,bottom=35mm,left=25mm,right=25mm,twoside]{geometry}
% pdftex
\usepackage{syntax}
\usepackage{ae}
\usepackage{amssymb}
\usepackage{setspace}
\usepackage{babel}
\usepackage{ifthen}
\usepackage{indentfirst}
\usepackage{longtable}
\usepackage{tabularx}
\usepackage[dvips]{graphicx}
\usepackage{subcaption}
%\usepackage[natbib=true,sorting=none, firstinits=true]{biblatex}
\usepackage[natbib=true, firstinits=true]{biblatex}
% \usepackage{cite} % should not be used with natbib
\usepackage{verbatim}
\usepackage{enumerate}
\usepackage{caption}
\usepackage{colortbl}
\usepackage{color}
\usepackage{textfit}
\usepackage{wrapfig}
\usepackage{bibentry}
\usepackage{hlundef}
\usepackage{slashbox}
\usepackage{multirow}
\usepackage[nottoc,notlof,notlot]{tocbibind}
\usepackage{url}                  %% usa pacote "url"
\usepackage{fancychap}
\usepackage{hyperref}
%\usepackage{pdfpages}
%\usepackage{titlesec}

\let\oldurl=\url
\def\url{\protect\oldurl}

\definecolor{cinza}{rgb}{0.5,0.5,0.5}

\let\cite=\citep
\clubpenalty=10000 \widowpenalty=10000 \exhyphenpenalty=10000
\hyphenpenalty=1000

\hypersetup{
  colorlinks=true,
  citecolor=black,
  filecolor=black,
  linkcolor=black,
  urlcolor=black
}

\nobibliography*
\newcommand{\apud}[2]{(\@apud#1,#2\@endapud)}
\def\@apud#1,#2\@endapud{%
   \citet{#1} \textbf{apud} \citet{#2}}%
%}

\newtheorem{definicao}{Defini��o}[chapter]
\newtheorem{teorema}{Teorema}[chapter]

\bibliography{referencias}

\begin{document}

\title{An�lise de Dados Eleitorais: \\
Uma abordagem baseada em visualiza��o}

\author{Francisco Morgani Fatore}

\titulation{\hyphenpenalty=10000 Monografia apresentada ao Instituto
de Ci�ncias Matem�ticas e de Computa��o -- ICMC/USP como parte dos requisitos para a obten��o do t�tulo de Bacharel na �rea de Ci�ncias de Computa��o e Matem�tica Computacional.} 
\advisor{Prof. Dr. Fernando Vieira Paulovich}
\address{USP -- S�o Carlos/SP}
\date{\ifcase\month\or
        Janeiro\or
        Fevereiro\or
        Mar�o\or
        Abril\or
        Maio\or
        Junho\or
        Julho\or
        Agosto\or
        Setembro\or
        Outubro\or
        Novembro\or
        Dezembro\fi/\the\year}

\begingroup
\maketitle

\frontmatter \pagestyle{plain}

\floatplacement{table}{!ht}

\input src/resumo.tex
%\input src/abstract.tex

\tableofcontents
\listoffigures
%\listoftables
%\input src/siglas.tex
\cleardoublepage{\pagestyle{plain}\clearpage}

\endgroup

\mainmatter

\renewcommand{\chaptermark}[1]{%

\markboth{\chaptername
\ \thechapter.\ #1}{}}  %

\renewcommand{\sectionmark}[1]{%
 \markright{\thesection.\ #1}}

\input src/intro.tex
\input src/revisao.tex
\input src/banco.tex
\input src/analises.tex
\input src/conclusao.tex

\printbibliography

\end{document}

















